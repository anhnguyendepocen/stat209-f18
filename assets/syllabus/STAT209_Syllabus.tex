\documentclass[12pt]{article}

\usepackage{fontspec}
\usepackage{geometry}
\usepackage{lastpage}
\usepackage{fancyhdr}
\usepackage{hyperref}

\geometry{top=1in, bottom=1in, left=1in, right=1in, marginparsep=4pt, marginparwidth=1in}

\renewcommand{\headrulewidth}{0pt}
\pagestyle{fancyplain}
\fancyhf{}
\lfoot{As of 2018-03-19}
\rfoot{page \thepage\ of \pageref{LastPage}}

\setlength{\parindent}{0pt}
\setlength{\parskip}{0pt}

% \setromanfont [Ligatures={Common}, Numbers={OldStyle}, Variant=01,
%  BoldFont={LinLibertine_RB.otf},
%  ItalicFont={LinLibertine_RI.otf},
%  BoldItalicFont={LinLibertine_RBI.otf}
%  ]{LinLibertine_R.otf}

\usepackage{tikz}
\def\checkmark{\tikz\fill[scale=0.4](0,.35) -- (.25,0) -- (1,.7) -- (.25,.15) -- cycle;}

\usepackage{xunicode}
\defaultfontfeatures{Mapping=tex-text}

\setromanfont{YaleNew}

\begin{document}

\begin{center}
{\bf MATH/STAT 209: Introduction to Statistical Modeling, Fall 2018} \\
Tuesday, Thursday 10:30-11:45 \quad JPSN G28\\
\end{center}

\bigskip

\noindent
\begin{tabular}{ l l }
{\bf Instructor:} &  {\bf Taylor Arnold} \\
E-mail: & \href{mailto:tarnold2@richmond.edu}{tarnold2@richmond.edu} \\
Office: & Jepson Hall, Rm 218 \\
Office hours: & To Be Determined\\
\end{tabular}

\vspace{0.5cm}

\textbf{Description:} \vspace{6pt}

This course broadly covers the entire process of collecting,
cleaning, visualizing, modeling, and presenting datasets. It
has a MATH designation but is not a \textit{mathematics} course.
The focus is on applied statistics and data analysis
rather than a detailed study of symbolic mathematics. By the end
of the semester you will feel confident collecting, analyzing,
and writing about datasets from a variety of fields. You will be
able to use these skills to address data-driven problems in a wide
range of application domains.

\bigskip

\textbf{Computing:} \vspace{6pt}

To facilitate your ability to actually \textit{do} statistics,
most class meetings will involve some form of computing.
No prior programming experience is assumed or required.

\medskip

We will use the \textbf{R} programming environment throughout the
semester. It is freely available for all major operating systems and
is pre-installed on many campus computers. You can download it and
all supporting files for your own machine via these links:
\begin{center}
\url{https://cran.r-project.org/} \\
\url{https://www.rstudio.com/}
\end{center}
The lab computers in Jepson are available and contain all of the
required software. I strongly recommend, however, downloading these
on your own machine so that you will be able to work on assignments
without needing to work only in the computer lab.

\bigskip

\textbf{Course Website:} \vspace{6pt}

All of the materials and assignments for the course will be posted
on the class website:
\begin{quote}
\url{https://statsmaths.github.io/stat209-f18}
\end{quote}
The website contains notes, assignment details, and
supplemental materials. At the end of the semester, this version of
the course will be archived and available for your reference.

\vspace{0.4cm}

\textbf{GitHub:} \vspace{6pt}

Your work for this semester will be submitted through GitHub,
the same platform that hosts our website, using the GitHub
classroom program. You will need to set up a free account, which
we will cover during the first several week of class.

\newpage

\textbf{Labs:} \vspace{6pt}

Nearly every course meeting will have an associated lab to complete. We will
usually work on these in class. By the start of the next course meeting, you
must complete the questions contained within the lab notebook. Assignments
will be submitted through GitHub; this process will explained in more detail
during class. \\

\bigskip

\textbf{Midterm Exams:} \vspace{6pt}

The course will have three in-class exams throughout the semester. These will
resemble the daily lab assignments and will be administered on the computers
in our classroom. Dates for these exams will be distributed roughly evenly
throughout the first twelve weeks of the course:
\begin{itemize}\setlength\itemsep{0em}
\item \textbf{Midterm I}: 2018-09-20 (Thursday)
\item \textbf{Midterm II}: 2018-10-23 (Tuesday)
\item \textbf{Midterm III}: 2018-11-15 (Thursday)
\end{itemize}
The two weeks of class following Thanksgiving break will be focused on
finishing and presenting your final data project.\\

\bigskip

\textbf{Data Projects:} \vspace{6pt}

While the midterms serve to make sure you are following along
with the general concepts, the ultimate aim of the course is to teach
you how to \textit{apply} statistics to real-world questions. To
this end, you will also complete a final data-oriented project for the
course. Details for this project and how it will be graded will be given
towards the middle of the term. The final week of the course will consist
of presentations of your findings.

\bigskip

\textbf{Final Grades:} \vspace{6pt}

The final grade will be determined by weighting the labs, exams, and project
as follows:
\begin{itemize}\setlength\itemsep{0em}
\item \textbf{Labs and Participation}: 20\%
\item \textbf{Midterm Exams}: 60\% (20\% each)
\item \textbf{Final Project}: 20\%
\end{itemize}
Course expectations and community standards will be discussed, developed and
distributed in the first week of the course. This will include policies for
class participation, attendance, and late work.

\bigskip

\textbf{Weekly Topics:} \vspace{6pt}

These are subject to change, but will give you a sense of what will be covered
throughout the semester:
\begin{itemize}\setlength\itemsep{0em}
\item Week 01: Introduction to Data Analysis % August 28, 30
\item Week 02: Introduction to R and Data Construction % September 4, 6 
\item Week 03: Reproducible Research and Variable Types % September 11, 13
\item Week 04: Review and MIDTERM I % September 18, 20
\item Week 05: A Grammar for Data Visualization % September 25, 27
\item Week 06: Annotating Graphics for Publication  % October 2, 4
\item Week 07: Filtering and Mutating Data % October 9, 11 
\item Week 08: Summarizing Numeric Data  % October 16 (Fall Break), 18
\item Week 09: Review and MIDTERM II % October 23, 25
\item Week 10: Statistical Inference %  % October 30, November 1
\item Week 11: Joining and Reshaping Data % November 6, 8
\item Week 12: Review and MIDTERM III % November 13, 15
%\item Week 13: Thanksgiving Break % November 20 (Thanksgiving)
\item Week 13: Final Project Consultations % November 27, 29 
\item Week 14: Final Project Presentations % December 4, 6
\end{itemize}

\textbf{Notice:} \vspace{6pt}

I reserve the right to modify this syllabus, with advanced warning, throughout
the semester. If necessary, I will email the class list and post an updated
version of the document on the course website.




% %%%%%%%%%%%%%%%%%%%%%%%%%%%%%%%%%%%%%%%%%%%%%%%%%%%%%%%%%%%%%%%%%%%%%
% \newpage

% \textbf{Learning Objectives (detail):} \vspace{6pt}

% Below is a more detailed description of each week's material. Note
% that the exact topics, pace, and coverage may change over the course
% of the semester.

% \vspace{1.0cm}

% \def\labelitemi{}
% \def\labelitemii{}

% \underline{Week 01, Reproducible research:}
% In this unit we explore the basics of statistical computing.
% We look at examples and benefits of plain text formats for data,
% code, and analyses.
% \begin{itemize}\setlength\itemsep{0em}
% \item
%   install R, RStudio and user-contributed packages
% \item
%   understand basic version control using the web-based GitHub GUI
% \item
%   create CSV files and read them into R
% \item
%   basic techniques for accessing variables in data objects
% \item
%   selecting variable names
% \item
%   constructing a data dictionary
% \end{itemize}

% \bigskip

% \underline{Week 02, Variable types and numeric summaries:}
% We begin our study of tabular data, with observations stored in rows and
% variables stored in columns. We start by describing the types of
% data that can be stored. Next, methods for summarizing and graphing
% numeric and categorical data are developed.
% \begin{itemize}\setlength\itemsep{0em}
% \item
%   describe and compute means and medians (using R and by hand)
% \item
%   describe and compute quantiles (using R and, for simple cases, by
%   hand)
% \item
%   describe and compute the standard deviation
% \item
%   differentiate between numeric and categorical data
% \item
%   treating numeric data as categorical data
% \item
%   create categorical variables by grouping numeric data
% \end{itemize}

% \bigskip

% \underline{Week 03, The Grammar of Graphics:}
% Building off of our basic plots, here we describe a self-contained
% system for the creation of statistical graphics. Putting the topics
% together allow for the construction of arbitrarily complex
% visualizations of data.
% \begin{itemize}\setlength\itemsep{0em}
% \item
%   the basic data aesthetics (x, y, label, color, size, and alpha)
% \item
%   the \textbf{ggplot2} syntax
% \item
%   mapping variables to aesthetics
% \item
%   setting fixed aesthetics
% \item
%   constructing layers of points, lines
% \end{itemize}

% \bigskip

% \underline{Week 04, Advanced Graphics:}
% Here we build off of the grammar of graphics to include
% other layer types, manual annotations, and building
% professional graphics.
% \begin{itemize}\setlength\itemsep{0em}
% \item
%   describe scales and themes in the grammar of graphics
% \item
%   use manual annotations to give context to graphics
% \item
%   make use of multiple datasets within a single plot
% \item
%   faceting by categorical variables
% \end{itemize}

% \bigskip

% \underline{Week 05, Filtering and Summarizing Data:}
% Often, data is given or found in a different format than is required for
% an analysis. In this unit, we study techniques for filtering and
% restructuring data. One particular focus is the study of how to
% change the \textit{level of analysis} of a data set, such as taking
% data originally about individuals and turning it into a dataset to
% study cities or counties.
% \begin{itemize}\setlength\itemsep{0em}
% \item
%   syntax of the filter command
% \item
%   boolean variables
% \item
%   binary operators: ``and'', ``or''
% \item
%   binary operators: greater than, less than
% \item
%   set containment
% \item
%   random subsets
% \item
%   syntax of the summarize command
% \item
%   counting grouped data
% \end{itemize}

% \bigskip

% \underline{Week 06, Statistical Inference:}
% Given a sample of data taken from a larger population, we can use
% models to estimate how well the sample resembles the entire population.
% This unit covers an introduction to these techniques, known as statistical
% inference. The same techniques can be used to study the outcome of random
% processes.
% \begin{itemize}\setlength\itemsep{0em}
% \item
%   sample and population statistics
% \item
%   independence
% \item
%   standard errors
% \item
%   confidence intervals
% \item
%   t-tests
% \end{itemize}

% \bigskip

% \underline{Week 07, Data Analysis and Review:}
% A review of the prior weeks and applications to a
% new dataset. First midterm on Thursday.

% \bigskip

% \underline{Week 08, Communicating Statistical Results:}
% In this unit we cover how to construct arguments using evidence derived
% from data.
% \begin{itemize}\setlength\itemsep{0em}
% \item
%   deductive versus inductive reasoning
% \item
%   understanding audience (technical vs.~general)
% \item
%   describe exploratory work and hypothesis / thesis generation
% \item
%   describe inferential statistics and hypothesis validation
% \item
%   include graphical annotations
% \end{itemize}

% \bigskip

% \underline{Spring Break}

% \bigskip

% \underline{Week 09, Normalized Data:}
% In this unit, we will explore and implement best practices for collecting and
% organizing data.
% \begin{itemize}\setlength\itemsep{0em}
% \item
%   determining table variables
% \item
%   specifying variable types
% \item
%   consistency standards
% \item
%   ISO date and time standards (ISO 8601)
% \item
%   standards for location data: country codes (ISO 3166), languages (ISO
%   639), currency (ISO 4217), and US FIPS codes
% \item
%   the tidy data model: rows, columns, and tables
% \end{itemize}

% \bigskip

% \underline{Week 10, Joining Relational Data:}
% We also build off of the last week's material by exploring
% concepts of data manipulation and data
% collection to describe methods for working simultaneously with many
% tables linked together by common keys.
% \begin{itemize}\setlength\itemsep{0em}
% \item
%   primary keys
% \item
%   foreign keys
% \item
%   composite keys
% \item
%   inner and outer joins
% \item
%   filtering joins
% \end{itemize}

% \bigskip

% \underline{Week 11, Working with strings:}
% Here, we study techniques for using manipulating data stored as strings. We use the stringi library in R to apply functions using fixed strings as well as a standard for describing patterns called regular expressions. You will become familiar with the following tasks and concepts:
% \begin{itemize}\setlength\itemsep{0em}
% \item detecting substrings
% \item extracting substrings
% \item removing substrings
% \item counting substrings
% \item describing repeating patterns
% \item denoting letters, numbers, and word boundaries
% \item anchoring regular expressions
% \item the UTF-8 encoding
% \item ICU and ISO-639
% \end{itemize}

% \bigskip

% \underline{Week 12, Text mining:}
% This unit builds off of the basic string processing tasks to study
% textual corpora. You will become familiar with the following concepts
% and comfortable applying them to new, raw textual data:
% \begin{itemize}\setlength\itemsep{0em}
% \item tokenization
% \item term-frequency matrices
% \item lemmatization
% \item part-of-speech tags
% \item dependencies
% \item named entities
% \end{itemize}

% \bigskip

% \underline{Week 13, Ethical Guidelines for Statistical Practice:}
% We discuss ethical issues surrounding the practice of data analysis.
% These include:
% \begin{itemize}\setlength\itemsep{0em}
% \item data ownership
% \item informed consent and IRB
% \item data privacy and anonymity
% \item open data
% \item $p$-hacking / data dredging
% \item problematic variables, discrimination, and proxies
% \item algorithmic fairness
% \item confirmation bias
% \end{itemize}

% \bigskip

% \underline{Week 14, Case Study and Review:}
% A review of the prior weeks and applications to a
% new dataset. Second midterm on Thursday.




\end{document}





