\documentclass[12pt]{article}

\usepackage{fontspec}
\usepackage{geometry}
\usepackage{lastpage}
\usepackage{fancyhdr}
\usepackage{hyperref}

\geometry{top=1in, bottom=1in, left=1in, right=1in, marginparsep=4pt, marginparwidth=1in}

\renewcommand{\headrulewidth}{0pt}
\pagestyle{fancyplain}
\fancyhf{}
\lfoot{As of 2018-08-11}
\rfoot{page \thepage\ of \pageref{LastPage}}

\setlength{\parindent}{0pt}
\setlength{\parskip}{0pt}

% \setromanfont [Ligatures={Common}, Numbers={OldStyle}, Variant=01,
%  BoldFont={LinLibertine_RB.otf},
%  ItalicFont={LinLibertine_RI.otf},
%  BoldItalicFont={LinLibertine_RBI.otf}
%  ]{LinLibertine_R.otf}

\usepackage{tikz}
\def\checkmark{\tikz\fill[scale=0.4](0,.35) -- (.25,0) -- (1,.7) -- (.25,.15) -- cycle;}

\usepackage{xunicode}
\defaultfontfeatures{Mapping=tex-text}

\setromanfont{YaleNew}

\begin{document}

\begin{center}
{\bf MATH/STAT 209: Introduction to Statistical Modeling, Fall 2018} \\
Tuesday, Thursday 10:30-11:45 \quad MRC LL1\\
\end{center}

\bigskip

\noindent
\begin{tabular}{ l l }
{\bf Instructor:} &  {\bf Taylor Arnold} \\
E-mail: & \href{mailto:tarnold2@richmond.edu}{tarnold2@richmond.edu} \\
Office: & Jepson Hall, Room 218
\end{tabular}

\vspace{0.5cm}

\textbf{Description:} \vspace{6pt}

This course broadly covers the entire process of collecting,
cleaning, visualizing, modeling, and presenting datasets. The goal is to
develop skills needed to engage in the ethical and insightful analysis of
data. This involves building technical proficiency, written and oral
communication skills, and the ability to think critically about open ended
tasks.

\medskip

Our focus will be on applied statistics and data analysis rather than a
detailed study of symbolic mathematics. It has a MATH designation but is not
a \textit{mathematics} course. By the end of the semester you will
feel confident collecting, analyzing, and writing about datasets
from a variety of fields. You will be able to use these skills to
address data-driven problems in a wide range of application domains.

\bigskip

\textbf{Course Website:} \vspace{6pt}

All of the materials and assignments for the course will be posted
on the class website:
\begin{quote}
\url{https://statsmaths.github.io/stat209-f18}
\end{quote}
The website contains notes, assignment details, and
supplemental materials. At the end of the semester, this version of
the course will be archived and available for your reference.

\bigskip

\textbf{Computing:} \vspace{6pt}

To facilitate your ability to actually \textit{do} statistics,
most class meetings will involve some form of computing.
No prior programming experience is assumed or required.

\medskip

We will use the \textbf{R} programming environment throughout the
semester. It is freely available for all major operating systems and
is pre-installed on many campus computers. You can download it and
all supporting files for your own machine via these links:
\begin{center}
\url{https://cran.r-project.org/} \\
\url{https://www.rstudio.com/}
\end{center}
The lab computers in Jepson are available and contain all of the
required software. You will, however, need to download these on your own
machine as this semester we are not meeting in a computer lab. If this poses
a challenge, please let me know during the first week of the semester.

\bigskip

\textbf{GitHub:} \vspace{6pt}

All of your work for this semester will be submitted through GitHub,
the same platform that hosts our website. You'll need to set up a free
account, which we will cover during the week of class.

\bigskip

\textbf{Labs:} \vspace{6pt}

Most class meetings will have an assignment called a `lab' associated with it.
These consist of a set of questions that must be answered with either small
snippets of code or short descriptive answers. Often we will start these during
class, though they may sometimes be just for your own practice. Your solutions
must be uploaded to your GitHub page prior to the start of the next class
meeting. Labs are graded on a Pass/Fail scale.

\bigskip

\textbf{Exams:} \vspace{6pt}

There will be three take-home midterm exams given throughout the semester.
These are open notes and resources, though you must complete them on your
own. The scheduled dates for the exams are (all dates are Thursdays):
\begin{itemize}\setlength\itemsep{0em}
\item \textbf{2018-09-20}
\item \textbf{2018-10-18}
\item \textbf{2018-11-15}
\end{itemize}
The exam will be made available starting at the normal start of class on the
date above; you will have 24-hours to complete it. The course has no final
exam.

\bigskip

\textbf{Data Projects:} \vspace{6pt}

The ultimate aim of the course is to teach you how to \textit{apply} statistics
to real-world questions. To this end, you will complete four data-oriented
projects during the course of the semester. Details for each of these will be
given in class. The first three projects correspond to the three midterm exams.

\bigskip

\textbf{Final Grades:} \vspace{6pt}

The final grade will be determined by weighting the labs and projects as
follows:
\begin{itemize}\setlength\itemsep{0em}
\item \textbf{Labs and Participation}: 10\%
\item \textbf{Midterm Exams}: 30\% (10\% each)
\item \textbf{Projects}: 60\% (15\% each)
\end{itemize}
To pass the course, you must also miss no more than four class meetings.
Attendance requires that you arrive on-time, complete any out of class
assignments for the day, and fully engage with the course material.
Failing to fulfil these attendance requirements may result in a failing
grade of `V' or a reduction to your final course average at the instructor's
sole discretion.

\bigskip

\textbf{Class Policies:} \vspace{6pt}

The following class policies address some of the most common
questions and concerns that students have. If anything is
unclear, please feel free to contact me for clarification at
any point in the semester.

\begin{itemize}\setlength\itemsep{0em}
\item \textbf{Academic honesty:} Cheating and plagiarism are grave scholarly
offenses and potential grounds
for expulsion; they are also a major barrier to your intellectual development.
You are expected to familiarize yourself with the entirety of the
University of Richmond’s Honor Code. If you are confused or unsure about
appropriate citation protocol or any other aspect of the Honor code,
please consult me before turning in an assignment.
\item \textbf{Special approval:} If you have special approval forms for extra
time on exams or any other circumstances I should know about, please speak
with me as early as possible so that we can best accommodate your needs.
\item \textbf{Late work:} You are expected to submit all work on-time. Late 
data analysis reports will be accepted after the due date with a full letter
grade deduction for each 24 hour period it is late (rounded up).
\item \textbf{Attendance:} You are expected to both attend and participate in most
class meetings. If you must be absent due to illness or other pressing
need, please let me know by email as soon as possible. A habit of arriving
late, failing to participate, or failing to accomplish any out of class assignments
is considered equivalent to an absence.
\item \textbf{Make-up work:} In instances where students have a valid excuse for
missing an assessment, please get in touch with me within 24-hours of missing
class to make alternative arrangements.
\item \textbf{Class conduct:} During class I expect you to refrain from checking
email, being on phones, or working on assignments for other classes.
\item \textbf{Computers:} I except you to bring a working laptop with R and 
RStudio installed. If this poses a challenge, please speak with me at the 
start of the semester or as new problems arise.
\item \textbf{Office hours}: Rather than fixed weekly office hours, I will 
provide blocks of open times to meet with me particularly focused around
project due dates. If you find me in my office, poke your head in and I
am usually happy to meet on the spot. Otherwise, please email me to make an
appointment so that we can chat. Please note that appointments should
be booked at least 24 hours ahead of time.
\item \textbf{Email:} I will also answer questions by email (it can, in fact,
be much faster than scheduling an appointment for small issues). During the
week, I aim to respond within 24 hours, with emails sent over the weekend
responded to by Monday morning. If your question involves code, please attach
your current lab or report as that will expedite my answering your question(s).
\end{itemize}

\bigskip

\textbf{Notice:} \vspace{6pt}

I reserve the right to modify this syllabus, with advanced warning, throughout
the semester. If necessary, I will email the class list and post an updated
version of the document on the course website.




% %%%%%%%%%%%%%%%%%%%%%%%%%%%%%%%%%%%%%%%%%%%%%%%%%%%%%%%%%%%%%%%%%%%%%
% \newpage

% \textbf{Learning Objectives (detail):} \vspace{6pt}

% Below is a more detailed description of each week's material. Note
% that the exact topics, pace, and coverage may change over the course
% of the semester.

% \vspace{1.0cm}

% \def\labelitemi{}
% \def\labelitemii{}

% \underline{Week 01, Reproducible research:}
% In this unit we explore the basics of statistical computing.
% We look at examples and benefits of plain text formats for data,
% code, and analyses.
% \begin{itemize}\setlength\itemsep{0em}
% \item
%   install R, RStudio and user-contributed packages
% \item
%   understand basic version control using the web-based GitHub GUI
% \item
%   create CSV files and read them into R
% \item
%   basic techniques for accessing variables in data objects
% \item
%   selecting variable names
% \item
%   constructing a data dictionary
% \end{itemize}

% \bigskip

% \underline{Week 02, Variable types and numeric summaries:}
% We begin our study of tabular data, with observations stored in rows and
% variables stored in columns. We start by describing the types of
% data that can be stored. Next, methods for summarizing and graphing
% numeric and categorical data are developed.
% \begin{itemize}\setlength\itemsep{0em}
% \item
%   describe and compute means and medians (using R and by hand)
% \item
%   describe and compute quantiles (using R and, for simple cases, by
%   hand)
% \item
%   describe and compute the standard deviation
% \item
%   differentiate between numeric and categorical data
% \item
%   treating numeric data as categorical data
% \item
%   create categorical variables by grouping numeric data
% \end{itemize}

% \bigskip

% \underline{Week 03, The Grammar of Graphics:}
% Building off of our basic plots, here we describe a self-contained
% system for the creation of statistical graphics. Putting the topics
% together allow for the construction of arbitrarily complex
% visualizations of data.
% \begin{itemize}\setlength\itemsep{0em}
% \item
%   the basic data aesthetics (x, y, label, color, size, and alpha)
% \item
%   the \textbf{ggplot2} syntax
% \item
%   mapping variables to aesthetics
% \item
%   setting fixed aesthetics
% \item
%   constructing layers of points, lines
% \end{itemize}

% \bigskip

% \underline{Week 04, Advanced Graphics:}
% Here we build off of the grammar of graphics to include
% other layer types, manual annotations, and building
% professional graphics.
% \begin{itemize}\setlength\itemsep{0em}
% \item
%   describe scales and themes in the grammar of graphics
% \item
%   use manual annotations to give context to graphics
% \item
%   make use of multiple datasets within a single plot
% \item
%   faceting by categorical variables
% \end{itemize}

% \bigskip

% \underline{Week 05, Filtering and Summarizing Data:}
% Often, data is given or found in a different format than is required for
% an analysis. In this unit, we study techniques for filtering and
% restructuring data. One particular focus is the study of how to
% change the \textit{level of analysis} of a data set, such as taking
% data originally about individuals and turning it into a dataset to
% study cities or counties.
% \begin{itemize}\setlength\itemsep{0em}
% \item
%   syntax of the filter command
% \item
%   boolean variables
% \item
%   binary operators: ``and'', ``or''
% \item
%   binary operators: greater than, less than
% \item
%   set containment
% \item
%   random subsets
% \item
%   syntax of the summarize command
% \item
%   counting grouped data
% \end{itemize}

% \bigskip

% \underline{Week 06, Statistical Inference:}
% Given a sample of data taken from a larger population, we can use
% models to estimate how well the sample resembles the entire population.
% This unit covers an introduction to these techniques, known as statistical
% inference. The same techniques can be used to study the outcome of random
% processes.
% \begin{itemize}\setlength\itemsep{0em}
% \item
%   sample and population statistics
% \item
%   independence
% \item
%   standard errors
% \item
%   confidence intervals
% \item
%   t-tests
% \end{itemize}

% \bigskip

% \underline{Week 07, Data Analysis and Review:}
% A review of the prior weeks and applications to a
% new dataset. First midterm on Thursday.

% \bigskip

% \underline{Week 08, Communicating Statistical Results:}
% In this unit we cover how to construct arguments using evidence derived
% from data.
% \begin{itemize}\setlength\itemsep{0em}
% \item
%   deductive versus inductive reasoning
% \item
%   understanding audience (technical vs.~general)
% \item
%   describe exploratory work and hypothesis / thesis generation
% \item
%   describe inferential statistics and hypothesis validation
% \item
%   include graphical annotations
% \end{itemize}

% \bigskip

% \underline{Spring Break}

% \bigskip

% \underline{Week 09, Normalized Data:}
% In this unit, we will explore and implement best practices for collecting and
% organizing data.
% \begin{itemize}\setlength\itemsep{0em}
% \item
%   determining table variables
% \item
%   specifying variable types
% \item
%   consistency standards
% \item
%   ISO date and time standards (ISO 8601)
% \item
%   standards for location data: country codes (ISO 3166), languages (ISO
%   639), currency (ISO 4217), and US FIPS codes
% \item
%   the tidy data model: rows, columns, and tables
% \end{itemize}

% \bigskip

% \underline{Week 10, Joining Relational Data:}
% We also build off of the last week's material by exploring
% concepts of data manipulation and data
% collection to describe methods for working simultaneously with many
% tables linked together by common keys.
% \begin{itemize}\setlength\itemsep{0em}
% \item
%   primary keys
% \item
%   foreign keys
% \item
%   composite keys
% \item
%   inner and outer joins
% \item
%   filtering joins
% \end{itemize}

% \bigskip

% \underline{Week 11, Working with strings:}
% Here, we study techniques for using manipulating data stored as strings. We use the stringi library in R to apply functions using fixed strings as well as a standard for describing patterns called regular expressions. You will become familiar with the following tasks and concepts:
% \begin{itemize}\setlength\itemsep{0em}
% \item detecting substrings
% \item extracting substrings
% \item removing substrings
% \item counting substrings
% \item describing repeating patterns
% \item denoting letters, numbers, and word boundaries
% \item anchoring regular expressions
% \item the UTF-8 encoding
% \item ICU and ISO-639
% \end{itemize}

% \bigskip

% \underline{Week 12, Text mining:}
% This unit builds off of the basic string processing tasks to study
% textual corpora. You will become familiar with the following concepts
% and comfortable applying them to new, raw textual data:
% \begin{itemize}\setlength\itemsep{0em}
% \item tokenization
% \item term-frequency matrices
% \item lemmatization
% \item part-of-speech tags
% \item dependencies
% \item named entities
% \end{itemize}

% \bigskip

% \underline{Week 13, Ethical Guidelines for Statistical Practice:}
% We discuss ethical issues surrounding the practice of data analysis.
% These include:
% \begin{itemize}\setlength\itemsep{0em}
% \item data ownership
% \item informed consent and IRB
% \item data privacy and anonymity
% \item open data
% \item $p$-hacking / data dredging
% \item problematic variables, discrimination, and proxies
% \item algorithmic fairness
% \item confirmation bias
% \end{itemize}

% \bigskip

% \underline{Week 14, Case Study and Review:}
% A review of the prior weeks and applications to a
% new dataset. Second midterm on Thursday.




\end{document}





